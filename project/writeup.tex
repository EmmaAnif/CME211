\documentclass{article}
\usepackage{algorithm2e}
\begin{document}
\setlength{\parindent}{0pt}
\setlength{\parskip}{1ex}

\begin{algorithm}[H]
 \SetAlgoLined
 \KwData{a CSR matrix, a vector, a guess of the solution, tolerance}
 \KwResult{solution of a CSR matrix vector equation}
 initialize $u_0$\;
 $r_0 = b - A$ $u_0$\;
 $L2normr0 = L2norm(r_0)$\;
 $p_0 = r_0$\;
 $niter = 0$\;
 \While{$(niter < nitermax)$}{
  niter = niter + 1\;
  $alpha = (r_n^T$ $r_n)/(p_n^T$ $A$ $p_n)$\;
  $u_{n+1} = u_n + alpha_n$ $p_n$\;
  $r_{n+1} = r_n - alpha_n$ $A$ $p_n$\;
  \If{($L2normr/L2norm0 < threshold)$}{
   break\;
  }
  $beta_n = (r_{n+1}^T$ $r_{n+1})/(r_n^T$ $r_n)$\;
  $p_{n+1} = r_{n+1} + beta_n$ $p_n$\;
 }
 \caption{Conjugate Gradient pseudo-code}
\end{algorithm}

In the implementation of CGSolver, I used 5 different functions which I defined
in matvecops.cpp\\
\textbf{L2norm} $-$ This was used to calculate the L2-norm of a vector.\\
\textbf{dotProduct} $-$ This was used to calculate the dot product of two vectors.\\
\textbf{matVecProduct} $-$ This was used to calculate the matrix vector product of a
CSR matrix and a vector.\\
\textbf{scalVecProduct} $-$ This was used to calculate the product of a vector and a
scalar.\\
\textbf{sum2Vec} $-$ This was used to get the sum of two vectors.

The use of these five functions greatly reduced the length of my code and made
debugging easier.

\end{document}
